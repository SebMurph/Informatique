%%%%%%%%%%%%%%%%%%%%%%%%%%%%%%% DOCUMENTCLASS %%%%%%%%%%%%%%%%%%%%%%%%
\documentclass[12pt,svgnames]{article}
%%%%%%%%%%%%%%%%%%%%%%%%%%%%%%%%%%%%%%%%%%%%%%%%%%%%%%%%%%%%%%%%%%%%%%%

%%%%% packages %%%%%%%%%%%%%%%%%%%%%
\usepackage[utf8]{inputenc}
\usepackage{sectsty}
\usepackage{lmodern}
\usepackage{enumitem}
\usepackage{calc}
\usepackage{tikz}
\usetikzlibrary{calc}
\usepackage{natbib}
\usepackage{amsthm}
\usepackage{titlesec}
\usepackage{newtxtext,newtxmath}
\usepackage{lipsum}% texte bidon
\usepackage{appendix}
\usepackage{pdfpages} 
\usepackage{url}
\usepackage[margin=2.5cm]{geometry}
\usepackage[T1]{fontenc}
\usepackage[utf8]{inputenc}
\usepackage{textcomp}     
\usepackage{gensymb}
\usepackage{tikz}
\usepackage{multirow}
\usepackage{wrapfig}
\usepackage{makecell}
\usepackage{fancyhdr}
\usepackage{setspace}
\usepackage{hyperref}
\usepackage{tocloft}
%\usepackage[french]{babel}
\def\labelitemi{$\bullet$}
\usepackage{comment}

%%%%%% needed in french mode to have nice bullets %%%%%%%%%%%%%%%%%%%
\renewcommand{\labelitemi}{$\bullet$}


%%% page layout and font %%%%%%%%%%%%%%%%%%%%%
\renewcommand\familydefault{\sfdefault} %san serif font
\setlength{\parindent}{0pt}
\interfootnotelinepenalty=10000
\onehalfspacing

%%% Dotfill %%%%%%%%%%%%%%%%%%%%%
\def\dotfill#1{\cleaders\hbox to #1{.}\hfill}
\newcommand\dotline[2][.5em]{\leavevmode\hbox to #2{\dotfill{#1}\hfil}}
\DeclareUnicodeCharacter{00B0}{\degree}
\expandafter\let\expandafter\pnewline\csname\string\ \endcsname


%%% hyperref setup %%%%%%%%%%%%%%%%%%%%%
\newcommand\myshade{85}
\colorlet{mylinkcolor}{violet}
\hypersetup{
 linkcolor  = mylinkcolor,
 citecolor  = mylinkcolor!\myshade!black,
%  urlcolor   = myurlcolor!\myshade!black,
 colorlinks = true,
 %colorurl= true,
}


%%% Nice section titles using tickz %%%%%%%%%%%%%%%%%%%%%
\setlength{\cftsecindent}{0pt}% Remove indent for \section
\setlength{\cftsubsecindent}{0pt}% Remove indent for \subsection
\setlength{\cftbeforesecskip}{2pt}
\definecolor{myblue}{RGB}{255,127,0}
\definecolor{airforceblue}{rgb}{0.36, 0.54, 0.66}

\newcommand{\sectionheader}[1]{%
    \begin{tikzpicture}
      \node[draw,color=blue!05,fill=blue!05,thick,right,inner sep=8pt] (sectiontitle) {\begin{minipage}{\linewidth-8pt*2}\color{blue}{#1}\end{minipage}};
      \fill[fill=gray!75] (sectiontitle.north west) -- ++(0,6pt) -- ++(2cm,0) -- +(0.2cm,-6pt) -- cycle;
    \end{tikzpicture}%
}

\newcommand{\subsectionheader}[1]{%
    \begin{tikzpicture}
      \node[draw,color=blue!05,fill=blue!05, right,inner sep=4pt] (sectiontitle) {\begin{minipage}{\linewidth-8pt*2}\color{black}{#1}\end{minipage}};
      \fill[fill=gray!75] (sectiontitle.north west) -- ++(0,6pt) -- ++(2cm,0) -- +(0.2cm,-6pt) -- cycle;
    \end{tikzpicture}%
}

\titleformat{\section}
  [hang]% style : hang, display, runin, leftmargin, ...
  {\large\bfseries\sffamily}% fonte numéro + titre
  {}% numéro
  {0em}% espace entre le numéro et le titre
  {\sectionheader}% fonte titre
\titlespacing*{\section}
  {0pt}% retrait à gauche
  {2em plus 0.3em minus .1em}% espace avant
  {0.5em}% espace après
  [0pt]% retrait à droite

\titleformat{\subsection}
  [hang]% style : hang, display, runin, leftmargin, ...
  {\normalfont\bfseries\sffamily}% fonte numéro + titre
  {\thesubsection}% numéro
  {1em}% espace entre le numéro et le titre
  {\subsectionheader}% fonte titre
\titlespacing*{\subsection}
  {0pt}% retrait à gauche
  {1em plus 0.3em minus .1em}% espace avant
  {0.1em }% espace après
  [0pt]% retrait à droite
\setcounter{secnumdepth}{0}


 %%%%%% headers and footer titles %%%%%%%%%%%%%%%%%%%%%%%%%
\pagestyle{fancy}
\fancyhf{}
\rhead{\textit{GymInf, Introduction aux systèmes informatiques}}
\lhead{\textit{S. Murphy}}
\fancyfoot[C]{\thepage}



%%%%%%%%%%%%%%%%%%%%%%%%%%%%%%% BEGIN DOCUMENT %%%%%%%%%%%%%%%%%%%%%%%%
\begin{document}
%%%%%%%%%%%%%%%%%%%%%%%%%%%%%%%%%%%%%%%%%%%%%%%%%%%%%%%%%%%%%%%%%%
%%%%%%%%%%%%%%%%%%%%%%%%%%%%%%%%%%%%%%%%%%%%%%%%%%%%%%%%%%%%%%%%%%


\begin{large}\textcolor{airforceblue}{\textbf{Introduction aux systèmes informatiques} \\
GymInf,  Année académique 2021-2022.}
\end{large}\\
\begin{Large}
\begin{center}
   \textcolor{airforceblue}{\textbf{Assistant vocal Echo Dot : composants d'input/output}}\\[.1cm]
\end{center}
\end{Large}
\textit{Sébastien Murphy}\\
\textit{Septembre 2021}\\
%\textit{nombres de caractères espaces incluses: 6442}
\\[.4cm]

%https://github.com/jhautry/echo-dot
%https://www.cambridge.org/core/services/aop-cambridge-core/content/view/9D6CFB31D939A9C4E04ABE8EC0DAE4CF/S2048770316000123a.pdf/signal-processing-and-analog-rf-circuit-design-cross-discipline-interactions-and-technical-challenges.pdf
%émetteur-récepteur

\textbf{Introduction.}
\textbf{Antennes}
\textbf{Comosants techniques}
MEDIATEK MT6625LN (BT 2.4 GHz, WIFI 2.4 GHz, GPS et FM sur un seul chip). BT wt WIFI sont emmetteur recepteurs. GPS et FM sont recepteurs.
%"Alexa met la radio". Le signal vocal est transmis chez Amazon.

%Sous système: réponse vocale.. / voice and action processing

%Alexa BT receiver: so it can function as a normal speaker

\textbf{Courte biographie: }J'ai ensuite travaillé un peu moins de deux ans en partenariat avec une entreprise basée à Neuchatel (G-Ray) qui faisait de la R\&D sur un nouveau concept de capteurs à Rayons-X ou l'absorbeur de photos (Germanium/ ou GaAs) était directement technologie de Wireless bonding. Des connaissances en electronique et dans le milieu médical.

\textbf{Le composant du echo dot choisi.}
Même ayant travaillé je n'ai pas de compréhension profonde du traitement du signal numérique. Important pour la culture générale à l'heure du déploemebt de la 5G et les adultes de demains vivront avec la 5G, la maison connécétée, les voitures autonomes et le signal sans fil fera partie integrante de leur quotidient. Pour pouvoir l'enseigner même à un niveau basique, il faut d'abord bien le comprendre.

Faire le bridge entre le monde analogique et digital à l'interieur de la machine. Traitement numérique du signal. Quelle est la différence fondamentales entre un signal Wifi et bluetooth, sont-ils échantilloné de la même façon? L'onde wifi peut transmettre jusqu'à 1 Gps de données, le BT 3 Mps\cite{}.  Pourquoi choisir de communiquer avec la maison par Bluetooth plutot que Wifi? Les deux signals ont la même fréquence centrée sur 2.4 GHz. Et plus généralement les aspects de traitement du signal Wifi/BT car les applications sont évidements énormes et ne font saisser de croitre. Directionalité versus Portée. On peut
Transmission des données à 20 Gps pour la 5G. Comprendre les enjeux de transmissions de données avec la 5G. 2.4 GHz.
D'une onde dans une gamme de fréquence donnée (anaéogique donc) à une information binaire.
Wifi: 1 Gps
4G: 1Gps
BT: 3 Mps.
Fréquence d'échantillonage

\bibliographystyle{apalike}
\bibliography{biblio}

\end{document}


